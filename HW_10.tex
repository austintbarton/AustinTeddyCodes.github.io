\documentclass[10pt,twoside]{article}
\usepackage{amssymb, amsmath, amsthm, amsfonts, epsfig, graphicx, dsfont,
  bbm, bbold, url, color, setspace, multirow, pinlabel}
\usepackage[all]{xy}

\usepackage{fancyhdr} \setlength{\voffset}{-1in}
\setlength{\topmargin}{0in} \setlength{\textheight}{9.5in}
\setlength{\textwidth}{6.5in} \setlength{\hoffset}{0in}
\setlength{\oddsidemargin}{0in} \setlength{\evensidemargin}{0in}
\setlength{\marginparsep}{0in} \setlength{\marginparwidth}{0in}
\setlength{\headsep}{0.25in} \setlength{\headheight}{0.5in}
\pagestyle{fancy}

\input{macros}

%--------------------------------------------------------%
\newcommand{\Pro}{\ensuremath{\mathbb{P}}}
%--------------------------------------------------------%

\onehalfspace

\fancyhead[LO,LE]{MATH 4317 - Professor Heil} \fancyhead[RO,RE]{Due 12/02/2022 at 11:59 pm}
\chead{\textbf{}} \cfoot{}
\fancyfoot[LO,LE]{} \fancyfoot[RO,RE]{Page \thepage\ of
  \pageref{LastPage}} \renewcommand{\footrulewidth}{0.5pt}
\parindent 0in
% ------------------------------------------------------%
% -------------------Begin Document---------------------%
% ------------------------------------------------------%
\begin{document}

\begin{center}
\huge{\bf{Homework (blank)} - Name}
\end{center}

\medskip

\noindent \large{\textbf{Collaborators: N/A}}

\medskip

\begin{itemize}
    \item\textbf{1) Problem 7.5.17} \newline
    \noindent\makebox[\linewidth]{\rule{18cm}{0.4pt}}
%\begin{prob} \label{sinx2_problem}
The function $h(x) = \sin(x^2)$ is continuous on $\R.$
Prove that it is not uniformly continuous on $\R.$
    \begin{proof}[Answer.]
    Let $f:\R\to\R$ be defined as $f(x) = x^2$ where $x\in \R$. Let $g:\R\to\R$ be defined as $g(x) = \sin(x)$.
    
    Then $g\circ f : \R\to\R$ is defined as $(g\circ f)(x) = g(f(x)) = \sin(x^2) = h(x)$. We will show that $f,g$ are both continuous on $\R$ and therefore $g\circ f = h$ is continuous on $\R$.
    
    First we will show that $f$ is continuous on $\R$. Let $\epsilon > 0$. Choose $0<\delta<1$ such that $\delta < \frac{\epsilon}{1 + 2|y|}$. Let $x\in \R$ and assume $y\in \R$ satisfies,
    \[  |x-y|<\delta  \]
    then,
    \[|f(x) - f(y)| = |x^2 - y^2| = |x-y||x+y|\]
    and since \[|x+y| = |x-y+2y|\leq |x-y|+2|y|<\delta + 2|y|<1 + 2|y|\]
    we have that
    \[|x-y||x+y|<\frac{\epsilon}{1 + 2|y|}(1 + 2|y|) = \epsilon\]
    That is,
    \[|f(x) - f(y)| < \epsilon\]
    Therefore, $f$ is continuous on $\R$.
    
    Now, we will show that $g$ is continuous on $\R$. Let $\epsilon > 0$. Let $\delta = \epsilon$. Let $x\in\R$ and assume $y\in \R$ satisfies,
    \[|x-y|<\delta\]
    then,
    \[|g(x) - g(y)| = |\sin(x) - \sin(y)| = 2\Big|\cos\biggparen{\frac{x+y}{2}}\Big|\Big|\sin\biggparen{\frac{x-y}{2}}\Big|\]
    where we used a trigonometric sum identity. But $\big|\cos\biggparen{\frac{x+y}{2}}\big| \leq 1$ and\newline $\big|\sin\biggparen{\frac{x-y}{2}}\big|\leq \big|\frac{x-y}{2}\big|$. Hence,
    \[2\Big|\cos\biggparen{\frac{x+y}{2}}\Big|\Big|\sin\biggparen{\frac{x-y}{2}}\Big| \leq 2\Big|\frac{x-y}{2}\Big| = |x-y| < \delta = \epsilon\]
    That is, 
    \[|g(x) - g(y)| < \epsilon\]
    Therefore, $g$ is continuous.\smallskip
    Since $f,g:\R\to\R$ are both continuous, then $g\circ f:\R\to\R$ is continuous on $\R$. And since $g\circ f = h$, we have that $h:\R\to\R$ defined as $h(x) = \sin(x^2)$ is continuous on $\R$.\medskip
    
    \rule{15cm}{1pt}
    
    To show that $h$ is not uniformly continuous, we will show that there exists $\epsilon > 0$ such that for all $\delta>0$ there exists $x,y\in\R$ such that
    \[ |x-y|  <\delta    \]
    and
    \[|h(x) - h(y)| \geq \epsilon\]\smallskip
    
    Let $\epsilon = 1$ and $n\in\N$. Choose $x = \sqrt{n\pi+\frac{\pi}{2}}$ and $y = \sqrt{n\pi}$. Clearly, $x,y\in\R$. Then for $n > \frac{1}{\delta^2}$, we have that
    \iftrue
    \begin{gather*}
        |x-y| = \Bigabs{\sqrt{n\pi+\frac{\pi}{2}} - \sqrt{n\pi}} = \sqrt{n\pi + \frac{\pi}{2}} - \sqrt{n\pi}
    \end{gather*}
    since $\sqrt{n\pi + \frac{\pi}{2}} >  \sqrt{n\pi}$ and, multiplying by $\frac{\sqrt{n\pi + \frac{\pi}{2}} + \sqrt{n}}{\sqrt{n\pi + \frac{\pi}{2}} + \sqrt{n}}$, gives us
    \begin{gather*}
        = \frac{n\pi + \frac{\pi}{2} - n\pi}{\sqrt{n\pi + \frac{\pi}{2}} + \sqrt{n}} = \frac{\frac{\pi}{2}}{\sqrt{n\pi + \frac{\pi}{2}} + \sqrt{n}}
    \end{gather*}
    and since $\sqrt{n\pi + \frac{\pi}{2}} >  \sqrt{n\pi}$, then $\sqrt{n\pi + \frac{\pi}{2}} + \sqrt{n\pi} > 2\sqrt{n\pi}$. Thus,
    \begin{gather*}
        \frac{\frac{\pi}{2}}{\sqrt{n\pi + \frac{\pi}{2}} + \sqrt{n}} < \frac{\frac{\pi}{2}}{2\sqrt{n\pi}} = \frac{\pi}{4\sqrt{n\pi}} < \frac{\pi}{\sqrt{n\pi}} = \frac{\sqrt{\pi}}{\sqrt{n}} < \frac{1}{\sqrt{n}}
    \end{gather*}
    that is, 
    \begin{gather*}
        |x-y| < \frac{1}{\sqrt{n}} < \delta
    \end{gather*}
    since $\delta > \frac{1}{\sqrt{n}}$ for $n> \frac{1}{\delta^2}$.
    \fi 
    But, 
    \begin{gather*}
        |h(x) - h(y)| = |\sin(x^2) - \sin(y^2)| = |\sin(n\pi + \frac{\pi}{2}) - \sin(n\pi)| = |1 - 0| = 1 \geq \epsilon
    \end{gather*}
    Note that since this is true for $n>\frac{1}{\delta^2}$, this is true for all $\delta > 0$ since the set of natural numbers is unbounded. Thus, no matter how small $\delta$ is, we may choose $n$ large enough so that $n>\frac{1}{\delta^2}$.\smallskip
    
    Therefore, we have that for $\epsilon = 1$ and for any $\delta >0$, there exists $x,y\in\R$ such that 
    
    \[|x-y|< \delta\] 
    and 
    \[|h(x) - h(y)| \geq \epsilon\]
    
    Thus, we have shown that $h$ is continuous but not uniformly continuous.
    
    \end{proof}
    
    \item\textbf{2) Problem 7.5.19} \newline
    \noindent\makebox[\linewidth]{\rule{18cm}{0.4pt}}
    Choose $a,$ $b > 0$ and define
$$f(t)
\Eq \begin{cases}
    |t|^a \tinierspace \sin(|t|^{-b}), & \text{if } t \ne 0, \\
    0, & \text{if } t = 0.
    \end{cases}$$
Prove that $f$ is Lipschitz on $(-1,1)$ if $a \ge 1+b.$
Is $f'$ continuous?

Hint: Since $f$ is even, it suffices to consider $t \ge 0.$
    \begin{proof}[Answer.]
    Assume $a\geq b+1$ where $a,b>0$. Notice that $f$ is even. That is, for $x\in (-1,1) $, \newline
    $f(-x) = f(x)$. Therefore, without loss of generality, we may choose $x\in(-1,1)$ such that $x\geq 0$. 
    
    Consider $x = 0$. Now, computing the derivative of $f$ at $x = 0$, 
    \begin{gather*}
        f'(x) = \lim_{h\to 0}\frac{f(x+h) - f(x)}{h} = \lim_{h\to 0}\frac{f(h)}{h}
    \end{gather*}
    since $h>0$, then $f(h) = h^a\sin(h^{-b})$, and we have,
    \begin{gather*}
        = \lim_{h\to 0}\frac{h^a\sin(h^{-b})}{h} = \lim_{h\to 0}h^{a-1}\sin(h^{-b}) = 0 \cdot 0 = 0
    \end{gather*}
    That is, for $x= 0$, then $f'(x) = 0$.\smallskip
    
    Consider $x>0$. Taking the derivative of $f$ at $x$, and using the product rule now, we have
    \begin{gather*}
        f'(x) = \ddx(x^a\sin(x^{-b})) = \ddx\big[x^a\big]\sin(x^{-b}) + x^a\ddx\big[\sin(x^{-b})\big]
    \end{gather*}
    Using the chain rule, $\ddx\big[\sin(x^{-b})\big] = \cos(x^{-b})\cdot (-b)x^{-b-1}$. And thus, we obtain,
    \begin{gather*}
        \ddx\big[x^a\big]\sin(x^{-b}) + x^a\ddx\big[\sin(x^{-b})\big] = ax^{a-1}\sin(x^{-b}) - bx^{a-b-1}\cos(x^{-b}) \\
        f'(x) = ax^{a-1}\sin(x^{-b}) - bx^{a-b-1}\cos(x^{-b})
    \end{gather*}
    Since $a\geq b+1$, then $a - b - 1\geq 0$ and $a - 1 \geq b > 0$. We know that $\cos,\sin$ are both bounded $-1,1$ and $x^{a-1}$ and $x^{1-b-1}$ are bounded over the set $[0,1)$. Also, clearly, $a,b$ are finite numbers.\smallskip
    
    Thus, $f'$ is bounded in either case and therefore bounded over $[0,1)$. Therefore, we can choose \newline
    $C =\displaystyle\sup_{x\in [0,1)}{|f'(x)|}$. By the mean value theorem, there is a $c\in [0,1)$ such that \[f'(c) = \frac{f(y) - f(x)}{y -x}\]
    which implies after taking absolute values and multiplying by the denominator,
    \[|f(x) - f(y)| = f'(c)|x - y| \leq C|x-y|\]
    
    Therefore, $f$ is Lipschitz on $(-1,1)$.\smallskip
    
    Additionally, $f'$ is continuous on $[0,1)$.
    
    \iffalse
    Thus, $x^{-b}$ is a strictly decreasing function and as $x$ approaches $0$, then $\cos(x^{-b})$ approaches $1$ and $\sin(x^{-b})$ approaches $0$.
    
    Additionally, $x^{a-1}$ is also a strictly increasing function that satisfies that $x^{a-1} < 1$ for all $0<x<1$ and $a - 1 > 0$. As $x$ approaches $1$, $x^{a-1}$ approaches $1$ and as $x$ approaches $0$, $x^{a-1}$ approaches $0$.
    
    Lastly, $x^{a-b-1}$ for $a-b-1>0$ is a strictly increasing function 
    \fi
    \end{proof}
    
    \item\textbf{3)  Problem 7.5.20} \newline
    \noindent\makebox[\linewidth]{\rule{18cm}{0.4pt}}
    (a) Suppose that $f \colon \R \to \R$ is continuous
and there is some $R > 0$ such that $f(x) = 0$ whenever $|x| > R.$
Prove that $f$ is uniformly continuous on $\R.$

Hint: Even though the domain of $f$ is not a compact set,
a theorem from class will be very helpful.

\medskip
(b) Suppose that a function $f \colon \R \to \R$ is continuous
and ``vanishes at infinity,'' i.e.,
$$\lim_{x \to \infty} f(x)
\Eq 0
\Eq \lim_{x \to -\infty} f(x).$$
Prove that $f$ is uniformly continuous on $\R.$

Remark: Part~(a) is a consequence of part~(b), so you could just do
part~(b) first and then say that part~(a) is a corollary.
But it might be insightful to think first about part~(a).

\medskip
(c) Give an example of a function that satisfies the hypotheses
of part~(b), but does not satisfy the hypotheses of part~(a).


    \begin{itemize}
    \pagebreak
        \item\begin{proof}[Answer to part (a)]
        We shall use the answer to part (b) to prove this. In part (b) we showed that if a function $f:\R\to\R$ is continuous and \[\lim_{x \to \infty} f(x)
        \Eq 0
        \Eq \lim_{x \to -\infty} f(x)\]
        then that $f$ is uniformly continuous on $\R$.\smallskip
        
        We will show that \[\lim_{x \to \infty} f(x)
        \Eq 0
        \Eq \lim_{x \to -\infty} f(x) = \lim_{|x|\to\infty} \Eq 0\]
        
        By definition of $f$, for all $x\in\R$ such that $|x|>R$, then $f(x) = 0$. Let $\epsilon >0$. Then for all $x$ such that $|x|>R$, we have that $f(x)= 0<\epsilon$. Therefore, as $|x|\to\infty$, then $\lim_{|x|\to\infty} \Eq 0$.\smallskip
        
        Therefore, using what we showed from part (b), it follows that $f$ is uniformly continuous on $\R$.
        
        \iffalse
        Let $K= \set{x\in\R:|x|\leq R}$ and $S = \set{x\in\R:|x|>R}$. Note that $K,S$ are disjoint sets and $K\cup S = \R$.\smallskip
        
        There are three general cases for $x,y\in \R$. Either $x,y\in K$, or $x,y\in S$, or, without loss of generality, $x\in K$ and $y\in S$.\smallskip
        
       % Additionally, $f$ is continuous on $\R$ and thus, continuous about the points $-R$ and $R$. Consider any $y\in S$. Then for all $\epsilon > 0$ there is a $\delta > 0$ such that if $|y - R| < \delta$ then $|f(y) - f(R)| < \epsilon$. But $f(y) = 0$ so that is $|f(R)|<\epsilon$.
        
        \textbf{Case 1:} $K$ is a closed subset of $\R$. Clearly, $K$ is also bounded specifically by $R$ by definition. By the Heine-Borel Theorem, $K$ is compact. Since $K$ is a compact subset of $\R$ and $f$ is continuous on $\R$, then $f$ is uniformly continuous on $K$.\smallskip
        
        \textbf{Case 2:} Consider $S$. By definition, for all $x,y\in S$, we have that $f(x) = 0$ and $f(y) = 0$. Choose any $C \geq 0$. Then we have that 
        \begin{gather*}
            |f(x) - f(y)| = 0\leq C|x-y|
        \end{gather*}
        for all $x,y\in S$. Therefore, $f$ is Lipschitz on $S$ and it follows that $f$ is uniformly continuous on $S$. \smallskip
        
        We know that $f$ is continuous on $\R$ and we have shown that $f$ is uniformly continuous on $K$ and $S$ each. 
        
        Let $\epsilon > 0$. Then, there exists $\delta > 0$ such that for any two points $x,z\in K$, if $|x - y|< \delta$ then $|f(x) - f(y)| < \frac{\epsilon}{2}$. 
        
        And for any two points $y,w\in S$, it follows that $|y-w|<\delta$ implies that \newline
        $|f(y) - f(w)| = 0$.
        
        %Additionally, $f$ is continuous on $\R$ so it is continuous about the points $-R$ and $R$. Therefore, for all $y\in \R$, there exists a $\delta > 0$ such that if $|R - y|< \delta$ then $|f(R) - f(y)| < \frac{\epsilon}{2}$. If $y\in S$, then $|f(R)| < \frac{\epsilon}{2}$. The same applies for $-R$. \smallskip
        
        \textbf{Case 3:} Let $x,y\in\R$ be such that $|x-y|< \delta$.
        
        Consider, without loss of generality, $x\in K$ and $y\in S$. Since $y\not\in K$ and $K$ is closed and nonempty, then by the closest point theorem, there exists a point $z\in K$ that is closer to $y$ than any other point in $K$. That is, 
        \[|y - z|\leq |x-y|\] for all $x\in K$. This point is $|z| = R$. So, $|y - R|\leq |x-y|$. And by definition of $K$, 
        \[|x|\leq R\] for all $x\in K$. Thus, since $R\in K$,
        \[|x - R|< |x - y| < \delta\]
        which implies that $|f(x) - f(R)| < \frac{\epsilon}{2}$ and thus,
        \begin{gather*}
            |f(x) - f(y)| \leq |f(x) - f(R)| + |f(R) - f(y)| \leq |f(x) - f(R)| + |f(R)| + |f(y)| \\
             < \frac{\epsilon}{2} + \frac{\epsilon}{2} = \epsilon
        \end{gather*}
        since $f$ is continuous on $\R$ and therefore continuous about the points $-R$ and $R$.\smallskip
        
        Therefore, we have shown that $f$ is uniformly continuous on $\R$.
        \fi
        
        \iffalse
        Consider, without loss of generality, $x\in K$ and $y\in S$. %Further, the closest that $x$ is to $S$ is $\dist(x, S)$. 
        Since $y\not\in K$ and $K$ is closed and nonempty, then by the closest point theorem, there exists a point $z\in K$ that is closer to $y$ than any other point in $K$. That is,
        \[|y - z| \leq |y-x|\]
        for all $x\in K$. For any $y\in S$, let $\delta_y = |y - z|$ where $z\in K$ and is closest to $y\not\in K = S$. Then, if,
        \[|x-y| < \delta_y = |y - z|\]
        then
        \[|f(x) - f(y)| = |f(x)| < |f(z)|\]
        \fi
        
        
        \iffalse
        For all $x,y\in K$, take $\displaystyle\sup_{x,y\in K}|f(x) - f(y)| = M$. This is, $M=\diam(f(K))$. We know such a value exists since  we showed that $K$ is compact and therefore $f(K)$ is compact and thus, $f(K)$ is bounded and has finite diameter. 
        \fi
        \end{proof}
        
        \newpage
        %----------------------------------------------------%
        
        \item\begin{proof}[Answer to part (b)]
        Let $\epsilon > 0$ and $x,y\in \R$. We will find a $\delta >0$ such that if $|x-y|<\delta$ then $|f(x) - f(y)|<\epsilon$.
        
        By definition of \[\lim_{x \to \infty} f(x)
        \Eq 0
        \Eq \lim_{x \to -\infty} f(x)\]
        there exists an $R>0$ such that for all $|x|>R$ we have that $|f(x)| < \frac{\epsilon}{3}$.
        
         Let $K = \set{x\in\R:|x|\leq R}$. Then $K$ is a closed subset of $\R$. Clearly, $K$ is also bounded specifically by $R$ by definition. By the Heine-Borel Theorem, $K$ is compact. Since $K$ is a compact subset of $\R$ and $f$ is continuous on $\R$, then $f$ is uniformly continuous on $K$. Therefore, there is a $\delta >0$ such that for $x,y\in K$ if $|x-y|<\delta$, then $|f(x) - f(y)|<\frac{\epsilon}{3}$.\smallskip
        
        There are three general cases for $x,y\in \R$. Either $|x|,|y|\leq R$, or $|x|,|y|> R$, or, without loss of generality, $|x|\leq R$ and $|y|> R$.\smallskip
        
        \textbf{Case 1:} $x,y\in K$. Then it follows that,
        \[|f(x) - f(y)| < \frac{\epsilon}{3} < \epsilon\]
        
        \textbf{Case 2:} Consider $|x|,|y|>R$. Then it follows that, from definition, $|f(x)| < \frac{\epsilon}{3}$ and $|f(y)| < \frac{\epsilon}{3}$ and thus,
        \[|f(x) - f(y)| < \frac{2\epsilon}{3}<\epsilon\]\smallskip
        
        \textbf{Case 3:} Let $|x|\leq R$ and $|y|>R$. We can assume without loss of generality that $y>R$. Then $x\leq R<y$ in this case. If we chose $y<-R$ then we would have that $y<-R\leq x$.
        
        Since $y\not\in K$ and $K$ is closed and nonempty, then by the closest point theorem, there exists a point $z\in K$ that is closer to $y$ than any other point in $K$. That is, 
        \[|y - z|\leq |x-y|\] for all $x\in K$. This point is $|z| = R$. So, $|y - R|\leq |x-y|$. And by definition of $K$, 
        \[|x|\leq R\] for all $x\in K$. Thus, since $R\in K$,
        \[|x - R|< |x - y| < \delta\]
        which implies that $|f(x) - f(R)| < \frac{\epsilon}{3}$ and thus,
        \begin{gather*}
            |f(x) - f(y)| \leq |f(x) - f(R)| + |f(R) - f(y)| \leq |f(x) - f(R)| + |f(R)| + |f(y)| \\
             < \frac{\epsilon}{3} + \frac{\epsilon}{3} + \frac{\epsilon}{3} = \epsilon
        \end{gather*}
        \smallskip
        
        Therefore, we have shown that $f$ is uniformly continuous on $\R$.
        \end{proof}
        
        \newpage
        
        \item\begin{proof}[Answer to part (c)]
        
        Consider the function $f:\R\to\R$ defined as
        \[f(x) = \frac{1}{x^2 + 1}\]
        
        \end{proof}
        
    \end{itemize}
    
    \pagebreak
    
    \item\textbf{4) Problem 7.5.21} \newline
    \noindent\makebox[\linewidth]{\rule{18cm}{0.4pt}}
    Let $f \colon \R \to \R$ be a function.
    Prove that $f$ is uniformly continuous on $\R$ if and only if
    $$\lim_{a \to 0} \, \biggparen{\sup_{x \in \R} \, |f(x) - f(x-a)|} \Eq 0.$$
    
    \begin{proof}[Answer.]
    Assume that $f$ is uniformly continuous on $\R$. Let $\epsilon > 0$. Then there exists $\delta > 0$ such that for all $x,y\in \R$ if
    \[|x-y|<\delta\]
    then 
    \[|f(x) - f(y)| < \epsilon\]
    Assume, without loss of generality, that $y\leq x$. Let $a\geq 0$ be such that $y = x-a$. Then, $0\leq a<\delta$ since,
    \[|x-y| = |x - (x-a)| = a < \delta\] and so,
    \[|f(x) - f(y)| = |f(x) - f(x-a)| < \epsilon\]
    Then, $\epsilon$ is an upper bound so the supremum $\displaystyle \sup_{x\in\R}|f(x) - f(x-a)|$ exists and,
    \[\sup_{x\in\R}\Bigabs{f(x) - f(x-a)} \leq \epsilon\]
    
    Note that clearly $0\leq \sup_{x\in\R}\Bigabs{f(x) - f(x-a)}$ for any $a$.
    
    Since $f$ is uniformly continuous, this applies for all $x,y\in\R$ and thus, for all $a\geq 0$. That is, for all $\epsilon > 0$, there exists a $\delta$ such that,
    
    \[|x - y| = |x - (x-a)| < \delta\]
    implies that
    \[\sup_{x\in\R}\Bigabs{f(x) - f(x-a)} \leq \epsilon\]
    and by definition of this supremum, this is for all points $x,y\in\R$ where $a = x - y$.
    
    Taking the limit as $a\to 0$, we have
    \[\lim_{a\to 0}\biggparen{\sup_{x\in\R}\Bigabs{f(x) - f(x-a)}}\leq \epsilon\]
    for all $\epsilon > 0$ since $\lim_{a\to 0}|x - (x-a)| = 0< \delta$ for all $\delta > 0$. Therefore,
    \[\lim_{a\to 0}\biggparen{\sup_{x\in\R}\Bigabs{f(x) - f(x-a)}} = 0\]\smallskip
    
    \rule{12cm}{0.5pt}
    
    Now, assume that 
    \[\lim_{a\to 0}\biggparen{\sup_{x\in\R}\Bigabs{f(x) - f(x-a)}} = 0\]
    Then, for all $\epsilon >0$, 
    \[\lim_{a\to 0}\biggparen{\sup_{x\in\R}\Bigabs{f(x) - f(x-a)}} \leq \epsilon\]
    So for any $a\geq 0$ there is some $\epsilon >0$ such that
    \[\sup_{x\in\R}\Bigabs{f(x) - f(x-a)} \leq \epsilon\]
    that is,
    \[|f(x) - f(x-a)|\leq \epsilon\]
    for all $x\in\R$.
    
    For any two points $x,y\in\R$, either $x<y$ or $y\leq x$.
    
    Without loss of generality, let $y\leq x$. Then for some $a\geq 0$, we have that $y = x - a$. Choose $\delta >a$. Then,
    \[|x - y| = |x - (x - a)| = a < \delta\]
    By assumption, $|f(x) - f(x-a)| = |f(x) - f(y)|\leq \epsilon$ for all $x\in \R$. That is, for every $\epsilon >0$ there exists $\delta >0$, such that if
    \[|x - y|<\delta\]
    then 
    \[|f(x) - f(y)|\leq \epsilon\]
    Therefore, $f$ is uniformly continuous.\medskip
    
    Thus, we have show that $f:\R\to\R$ is uniformly continuous if and only if 
    \[\lim_{a \to 0} \, \biggparen{\sup_{x \in \R} \, |f(x) - f(x-a)|} \Eq 0.\]
    \end{proof}
    
    \newpage
    
    \item\textbf{5) Problem 8.2.4} \newline
    \noindent\makebox[\linewidth]{\rule{18cm}{0.4pt}}
    Prove the following statements.

\smallskip
(a) The function $\norm{\cdot}_\uni$ defined
in equation (8.2) %\eqref{altintervaluninormdef_eq}
is a norm on $\Fc_b[0,1].$

Remark: It follows (you do not need to prove this) that
$\norm{\cdot}_\uni$ is also a norm $C[0,1],$
because this is a subspace of $\Fc_b[0,1].$

\medskip
(b) The function $\norm{\cdot}_1$ defined in
equation (8.5) %\eqref{altCabL1_eq}
is a norm on $C[0,1].$

Remark: You can use properties of integrals from Calculus without proof.
    \begin{itemize}
        \item\begin{proof}[Answer to part (a)]
        Let $f\in\Fc_b[0,1]$. By definition, $f$ must be a bounded scalar valued function whose domain is on the closed interval $[0,1]$. Since $f$ is bounded, there exists an $M>0$ such that $|f(t)|\leq M$ for all $t\in[0,1]$. 
        
        Thus, $||f||_u = \displaystyle\sup_{t\in[0,1]}|f(t)| \leq M<\infty$.
        
        Further, by definition of the uniform norm and the absolute value of a scalar value, we know that $0\leq ||f||_u$.
        
        Therefore, for any $f\in\Fc_b[0,1]$ we have that 
        \[0\leq ||f||_u<\infty\]
        which shows that the uniform norm on $\Fc_b[0,1]$ satisfies the non-negativity property of a norm.\smallskip
        
        Let $c\in\R$ and consider, $cf$. Then $$||cf|| = \displaystyle\sup_{t\in[0,1]}|cf(t)| = |c|\displaystyle\sup_{t\in[0,1]}|f(t)| = |c|||f||_u$$ where we used properties of the supremum (Section 3.4).
        
        Therefore, the uniform norm on $\Fc_b[0,1]$ satisfies the homogeneity property of a norm.\smallskip
        
        Let $g\in\Fc_b[0,1]$. Consider, \begin{gather*}
             ||f + g||_u = \sup_{t\in[0,1]}|(f+g)(t)| = \sup_{t\in[0,1]}|f(t)+g(t)| \leq \sup_{t\in[0,1]}|f(t)| + \sup_{t\in[0,1]}|g(t)| \\
             = ||f||_u + ||g||_u
        \end{gather*}
        where we used properties of the supremum. Thus,
        \[||f + g||_u\leq ||f||_u + ||g||_u\]
        which shows that the uniform norm on $\Fc_b[0,1]$ satisfies the triangle inequality property of a norm.\smallskip
        
        Assume, $||f||_u = 0$. Then that is, 
        \[\sup_{t\in[0,1]}|f(t)| = 0\]
        by definition, this means that, for all $t\in[0,1]$, we have that $|f(t)|\leq0$. But $0\leq |f(t)|$ for all $t$. Thus, $|f(t)| = f(t) = 0$ for all $t$. By definition, $f = 0$. That is, $f$ is the zero function denoted as $0\in\Fc_b[0,1]$.\smallskip
        
        Assume $f = 0$. Consider, $||f||_u$. Since $f(t) = 0$ for all $t\in[0,1]$, then $f(t)\leq 0$ for all $t$ and thus,
        \[0 \leq \sup_{t\in[0,1]}|f(t)| \leq 0\]
        and therefore, 
        \[||f||_u = \sup_{t\in[0,1]}|f(t)| = 0\]
        Therefore, we have shown that the uniform norm on $\Fc_b[0,1]$ satisfies the uniqueness property of a norm.\medskip
        
        Therefore, we have shown that the uniform norm is a norm on $\Fc_b[0,1]$.
        
        \end{proof}
        
        \item\begin{proof}[Answer to part (b)]
        
        Let $f\in C[0,1]$. Consider, $||f||_1$. By definition of the absolute value function and the Riemann integral within the norm,
        \[||f||_1 = \int_{0}^1 |f(t)|dt \geq 0\]
        Therefore, $||\cdot||_1$ satisfies the non-negativity property of a norm.\smallskip
        
        Let $c\in \R$. Consider,
        \begin{gather*}
            ||cf||_1 = \int_{0}^1 |cf(t)|dt = |c|\int_{0}^1 |f(t)|dt = |c||f||_1
        \end{gather*}
        Therefore, $||\cdot||_1$ satisfies the homogeneity property of a norm.\smallskip
        
        Let $g\in C[0,1]$. Consider,
        \begin{gather*}
            ||f + g||_1 = \int_0^1 |(f+g)(t)|dt = \int_{0}^1 |f(t) + g(t)|dt = \int_0^1 |f(t)| + |g(t)|dt \\
            = \int_0^1|f(t)| + \int_0^1 |g(t)|dt = ||f||_1 + ||g||_1
        \end{gather*}
        Therefore, $||\cdot||_1$ satisfies the triangle inequality property of a norm.\smallskip
        
        Assume $||f||_1 = 0$. Then,
        \[\int_0^1 |f(t)|dt = 0\]
        but $|f(t)|\geq 0$ for all $t\in [0,1]$. Therefore, it must be that $|f(t)| = f(t) = 0$ for all $t\in[0,1]$ which means that $f = 0$ where $0$ is the zero function $0\in C[0,1]$.\smallskip
        
        Assume that $f = 0$. Then $f(t) = |f(t)| =  0$ for all $t\in [0,1]$. Thus, 
        \[\int_{0}^1 |f(t)|dt = ||f||_1 = 0\]
        Therefore, we have shown that $||\cdot||_1$ satisfies the uniqueness property of a norm.\medskip
        
        Therefore, we have shown that $||\cdot||_1$ is a norm on $C[0,1]$.
        
        \end{proof}
        
    \end{itemize}
\end{itemize}

\label{LastPage}
\end{document}